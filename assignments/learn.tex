% STYLE SUGGESTIONS
%
% When showing steps in computations, line up equal signs vertically, one per line.
% Indent your code to make it more readable (and editable).
% Show relevant (calculus & diffeg) steps.
% Interpret solution and steps in context of the problem & class.
% Communicate your thought process so the reader does not have to assume or guess at what you think.
%

\documentclass{article}

\usepackage{amsmath}
\usepackage{amssymb}
\usepackage{graphicx}

\begin{document}
\title{Assignment 01}
\author{Ross Grinvalds}
\maketitle

Hello world!

\begin{enumerate}
	\item item one

	\item item two

	\begin{enumerate}
		\item[69.] best sublist
		
		\item[420.] ever.
	
	\end{enumerate}

	\item some math $x^{420}_{69}$
	
	\item
		\begin{align}
			0  &= x + y \\
				 &= soup  \label{soup} \\
				 &= s(oup) \label{soup_factor} \\
				 & \therefore \\
			soup	&= tonuts
		\end{align}
	
	\item In step (\ref{soup}), ...

	\item[a.]
		\begin{align*}
			y &= P^\intercal{}e_i \\
			&=\lambda_i
		\end{align*}

	\item[matrix example:]
		$\begin{bmatrix}
			1 & 2 & 3 \\
			a & b & c \\
		\end{bmatrix}$
		$$\int_{a}^{x}g(x)dt$$

	\item[table example:] Let's take the function $f(x) = \sqrt{x^2 + 1}$ and make a table of values.
		\begin{tabular}{r|cccc}
			$x$ & -1 & 0 & 1 & 2 \\
			\hline
			$f(x)$ & $\sqrt{2}$ & 1 & $\sqrt{2}$ & $\sqrt{5}$
		\end{tabular}

\end{enumerate}

\includegraphics{01/plots/Rplot.png}
\
\end{document}
